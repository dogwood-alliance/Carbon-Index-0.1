% Options for packages loaded elsewhere
\PassOptionsToPackage{unicode}{hyperref}
\PassOptionsToPackage{hyphens}{url}
%
\documentclass[
]{article}
\usepackage{lmodern}
\usepackage{amssymb,amsmath}
\usepackage{ifxetex,ifluatex}
\ifnum 0\ifxetex 1\fi\ifluatex 1\fi=0 % if pdftex
  \usepackage[T1]{fontenc}
  \usepackage[utf8]{inputenc}
  \usepackage{textcomp} % provide euro and other symbols
\else % if luatex or xetex
  \usepackage{unicode-math}
  \defaultfontfeatures{Scale=MatchLowercase}
  \defaultfontfeatures[\rmfamily]{Ligatures=TeX,Scale=1}
\fi
% Use upquote if available, for straight quotes in verbatim environments
\IfFileExists{upquote.sty}{\usepackage{upquote}}{}
\IfFileExists{microtype.sty}{% use microtype if available
  \usepackage[]{microtype}
  \UseMicrotypeSet[protrusion]{basicmath} % disable protrusion for tt fonts
}{}
\makeatletter
\@ifundefined{KOMAClassName}{% if non-KOMA class
  \IfFileExists{parskip.sty}{%
    \usepackage{parskip}
  }{% else
    \setlength{\parindent}{0pt}
    \setlength{\parskip}{6pt plus 2pt minus 1pt}}
}{% if KOMA class
  \KOMAoptions{parskip=half}}
\makeatother
\usepackage{xcolor}
\IfFileExists{xurl.sty}{\usepackage{xurl}}{} % add URL line breaks if available
\IfFileExists{bookmark.sty}{\usepackage{bookmark}}{\usepackage{hyperref}}
\hypersetup{
  pdftitle={Forest Carbon Index Report},
  hidelinks,
  pdfcreator={LaTeX via pandoc}}
\urlstyle{same} % disable monospaced font for URLs
\usepackage[margin=1in]{geometry}
\usepackage{color}
\usepackage{fancyvrb}
\newcommand{\VerbBar}{|}
\newcommand{\VERB}{\Verb[commandchars=\\\{\}]}
\DefineVerbatimEnvironment{Highlighting}{Verbatim}{commandchars=\\\{\}}
% Add ',fontsize=\small' for more characters per line
\usepackage{framed}
\definecolor{shadecolor}{RGB}{248,248,248}
\newenvironment{Shaded}{\begin{snugshade}}{\end{snugshade}}
\newcommand{\AlertTok}[1]{\textcolor[rgb]{0.94,0.16,0.16}{#1}}
\newcommand{\AnnotationTok}[1]{\textcolor[rgb]{0.56,0.35,0.01}{\textbf{\textit{#1}}}}
\newcommand{\AttributeTok}[1]{\textcolor[rgb]{0.77,0.63,0.00}{#1}}
\newcommand{\BaseNTok}[1]{\textcolor[rgb]{0.00,0.00,0.81}{#1}}
\newcommand{\BuiltInTok}[1]{#1}
\newcommand{\CharTok}[1]{\textcolor[rgb]{0.31,0.60,0.02}{#1}}
\newcommand{\CommentTok}[1]{\textcolor[rgb]{0.56,0.35,0.01}{\textit{#1}}}
\newcommand{\CommentVarTok}[1]{\textcolor[rgb]{0.56,0.35,0.01}{\textbf{\textit{#1}}}}
\newcommand{\ConstantTok}[1]{\textcolor[rgb]{0.00,0.00,0.00}{#1}}
\newcommand{\ControlFlowTok}[1]{\textcolor[rgb]{0.13,0.29,0.53}{\textbf{#1}}}
\newcommand{\DataTypeTok}[1]{\textcolor[rgb]{0.13,0.29,0.53}{#1}}
\newcommand{\DecValTok}[1]{\textcolor[rgb]{0.00,0.00,0.81}{#1}}
\newcommand{\DocumentationTok}[1]{\textcolor[rgb]{0.56,0.35,0.01}{\textbf{\textit{#1}}}}
\newcommand{\ErrorTok}[1]{\textcolor[rgb]{0.64,0.00,0.00}{\textbf{#1}}}
\newcommand{\ExtensionTok}[1]{#1}
\newcommand{\FloatTok}[1]{\textcolor[rgb]{0.00,0.00,0.81}{#1}}
\newcommand{\FunctionTok}[1]{\textcolor[rgb]{0.00,0.00,0.00}{#1}}
\newcommand{\ImportTok}[1]{#1}
\newcommand{\InformationTok}[1]{\textcolor[rgb]{0.56,0.35,0.01}{\textbf{\textit{#1}}}}
\newcommand{\KeywordTok}[1]{\textcolor[rgb]{0.13,0.29,0.53}{\textbf{#1}}}
\newcommand{\NormalTok}[1]{#1}
\newcommand{\OperatorTok}[1]{\textcolor[rgb]{0.81,0.36,0.00}{\textbf{#1}}}
\newcommand{\OtherTok}[1]{\textcolor[rgb]{0.56,0.35,0.01}{#1}}
\newcommand{\PreprocessorTok}[1]{\textcolor[rgb]{0.56,0.35,0.01}{\textit{#1}}}
\newcommand{\RegionMarkerTok}[1]{#1}
\newcommand{\SpecialCharTok}[1]{\textcolor[rgb]{0.00,0.00,0.00}{#1}}
\newcommand{\SpecialStringTok}[1]{\textcolor[rgb]{0.31,0.60,0.02}{#1}}
\newcommand{\StringTok}[1]{\textcolor[rgb]{0.31,0.60,0.02}{#1}}
\newcommand{\VariableTok}[1]{\textcolor[rgb]{0.00,0.00,0.00}{#1}}
\newcommand{\VerbatimStringTok}[1]{\textcolor[rgb]{0.31,0.60,0.02}{#1}}
\newcommand{\WarningTok}[1]{\textcolor[rgb]{0.56,0.35,0.01}{\textbf{\textit{#1}}}}
\usepackage{graphicx}
\makeatletter
\def\maxwidth{\ifdim\Gin@nat@width>\linewidth\linewidth\else\Gin@nat@width\fi}
\def\maxheight{\ifdim\Gin@nat@height>\textheight\textheight\else\Gin@nat@height\fi}
\makeatother
% Scale images if necessary, so that they will not overflow the page
% margins by default, and it is still possible to overwrite the defaults
% using explicit options in \includegraphics[width, height, ...]{}
\setkeys{Gin}{width=\maxwidth,height=\maxheight,keepaspectratio}
% Set default figure placement to htbp
\makeatletter
\def\fps@figure{htbp}
\makeatother
\setlength{\emergencystretch}{3em} % prevent overfull lines
\providecommand{\tightlist}{%
  \setlength{\itemsep}{0pt}\setlength{\parskip}{0pt}}
\setcounter{secnumdepth}{-\maxdimen} % remove section numbering

\title{Forest Carbon Index Report}
\author{}
\date{\vspace{-2.5em}}

\begin{document}
\maketitle

\begin{Shaded}
\begin{Highlighting}[]
\CommentTok{\# The \textasciigrave{}params\textasciigrave{} object is available in the document.}
\NormalTok{params}\OperatorTok{$}\NormalTok{stateText}
\end{Highlighting}
\end{Shaded}

\begin{verbatim}
## [1] "NA"
\end{verbatim}

\begin{Shaded}
\begin{Highlighting}[]
\KeywordTok{head}\NormalTok{(params}\OperatorTok{$}\NormalTok{data)}
\end{Highlighting}
\end{Shaded}

\begin{verbatim}
## [1] "NA"
\end{verbatim}

The great state of NA has \textless{}\textgreater{} total acres of
forestland, which store \textless{}\textgreater{} tons of carbon, for a
total Forest Carbon Index (FCI) of \textless{}\textgreater.

\hypertarget{forest-types}{%
\section{Forest Types}\label{forest-types}}

NA has \textless{}\textgreater{} forest types across the state, with
\textless{}\textgreater\% of acres being \textless{}\textgreater. There
are only \textless{}\textgreater{} acres of \textless{}\textgreater. The
\textless{}\textgreater{} has the highest FCI at
\textless{}\textgreater{} while the \textless{}\textgreater{} has the
lowest FCI at \textless{}\textgreater. Overall, the average FCI in NA is
\textless{}\textgreater.

\begin{Shaded}
\begin{Highlighting}[]
\NormalTok{stSumFt<{-}}\StringTok{ }\ControlFlowTok{function}\NormalTok{(data)\{}
  \KeywordTok{distinct}\NormalTok{(data}\OperatorTok{$}\NormalTok{Forest.type.group)}
\NormalTok{\}}
\CommentTok{\#Graph: Forest Types Pie Chart by Ac and by tC}
\end{Highlighting}
\end{Shaded}

\textless\textless\textless\textless\textless\textless\textless{} HEAD
\# Fake Forests

On average, natural forests in NA store \textless\textless StFCNatural -
StFCIPlanted\textgreater\textgreater{} additional tons of carbon per
acre than artificially regenerated forests. Across the state, there are
\textless{}\textgreater{} acres of plantations, representing
\textless{}\textgreater{} / \textgreater{} \% of total forestland.

If those acres of planted forests had been naturally regenerated, they
would currently be storing approximately \textless{}\textgreater{} =
((\textless{}\textgreater{} * \textless{}\textgreater) -
\textless{}\textgreater) more tons of carbon, equivalent to an
additional \textless{}\textgreater{} * 44/12 = CO2e * 0.192608384
passenger vehicles on the road for one year.

\begin{Shaded}
\begin{Highlighting}[]
\CommentTok{\#Graph: Forest Type X Origin, FCI}
\end{Highlighting}
\end{Shaded}

\hypertarget{how-does-na-compare}{%
\section{How Does NA Compare?}\label{how-does-na-compare}}

In the fourteen state region of the South, NA ranks:

*\textless{}\textgreater{} of 14 in Acres of Forestland

\begin{itemize}
\item
  \textless{}\textgreater{} of 14 in Tons of Carbon Sequestered in
  Forestland
\item
  \textless{}\textgreater{} of 14 in Forest Carbon Index (FCI)
\end{itemize}

Here's how the forest types of NA rank against other states:

Table: FCI Value, Forest Type Row, State Column

Table: TC Value, Forest Type Row, State Column

Table: AC Value, Forest Type Row, State Column

\hypertarget{natural-climate-solutions}{%
\section{Natural Climate Solutions}\label{natural-climate-solutions}}

\hypertarget{letting-forests-mature}{%
\subsection{Letting Forests Mature}\label{letting-forests-mature}}

\textbf{WHY}

Planted forests (plantations) usually have a short ``rotation'' -- the
number of years it takes for them to grow, be cut down, and have the
area replanted. In commercial pine plantations, it can be as little as
20-25 years. In contrast, natural forests often remain uncut for longer,
naturally store more carbon, and have the potential to transform into
old growth forests that support more biodiversity and provide more
ecosystem services.

Every time an acre of forest is harvested, it releases carbon. If that
acre is used for bioenergy, the carbon can be emitted in as little as
1-2 years. If that acre is used for more traditional forest products,
like paper, pulp, or lumber, 86\% of that carbon is emitted within 100
years, with just 14\% remaining behind in finished products and
landfills. Although plantations are generally intended for commercial
harvest, a substantial amount of natural forests also get harvested
every year.

Contrary to common belief, older forests do not stop or significantly
slow carbon sequestration. Instead, older forests continue to store
carbon well beyond 100 years of age. In fact, the carbon uptake rate is
remarkably similar for both younger and older forests.

\textbf{THE PAYOFF}

Reducing the amount of forest harvest in NA could improve carbon
sequestration and contribute to the state's natural climate solutions.
In the data, this would be observed as allowing more forests to mature
to the next age class. We can model this behavior by doing some simple
table calculations, and predict what the increase in carbon would be if
acres managed to successfully grow into the next age class.

\begin{itemize}
\item
  If the state moved 10\% of all forests to the next age class, it would
  result in an additional \textless{}\textgreater{} tons of carbon
  sequestered in NA's forests.
\item
  If the state moved 25\% of all forests to the next age class, it would
  result in an additional \textless{}\textgreater{} tons of carbon
  sequestered in NA's forests.
\item
  If the state moved 50\% of all forests to the next age class, it would
  result in an additional \textless{}\textgreater{} tons of carbon
  sequestered in NA's forests.
\end{itemize}

\begin{Shaded}
\begin{Highlighting}[]
\CommentTok{\#Interactive Table: Current State, Value: Carbon, Row: Forest Type Group, Col: Age Class}
\CommentTok{\#   Percentage slider 0{-}100. Takes X \% of previous age class and sticks it in the next.}

\CommentTok{\#Flat Table For Report: Benchmarks of 10, 25, 50\%, same table output}
\end{Highlighting}
\end{Shaded}

\hypertarget{letting-forests-regenerate-naturally}{%
\subsection{Letting Forests Regenerate
Naturally}\label{letting-forests-regenerate-naturally}}

\textbf{WHY}\\
Although planting trees does regenerate the area more quickly, there are
lasting effects from those plantings that are seen in carbon
sequestration. On average, natural forests in NA store
\textless\textless StFCNatural - StFCIPlanted\textgreater\textgreater{}
additional tons of carbon per acre than artificially regenerated
forests. This means that a naturally regenerated forest stand will, as a
rule, store more carbon than an artificially regenerated (planted)
stand. Beyond carbon benefits, natural regeneration also reduces the
amount of pesticides, soil impaction, and water runoff that can occur
after a forest harvest.

\textbf{THE PAYOFF} Although we cannot choose individual acres, we can
see the potential carbon sequestration difference if more acres were
naturally instead of artificially regenerated.

\begin{itemize}
\item
  If the state had 10\% more naturally regenerated forests, its forests
  would sequester \textless{}\textgreater{} more tons of carbon.
\item
  If the state had 25\% more naturally regenerated forests, its forests
  would sequester \textless{}\textgreater{} more tons of carbon.
\item
  If the state had 50\% more naturally regenerated forests, its forests
  would sequester \textless{}\textgreater{} more tons of carbon.
\end{itemize}

\begin{Shaded}
\begin{Highlighting}[]
\CommentTok{\#Interactive Table: Current State, Value: Carbon, Row: Forest Type Group, Col: Stand Origin}
\CommentTok{\#   Percentage slider 0{-}100. Takes X \% of artificial and changes it to natural.}

\CommentTok{\#Flat Table For Report: Benchmarks of 10, 25, 50\%, same table output}
\end{Highlighting}
\end{Shaded}

\hypertarget{risks-issues}{%
\subsection{Risks \& Issues}\label{risks-issues}}

Changing the status quo always comes with trade-offs. Restricting the
ability of the forest products industry to replant in commercially
favorable species will be unpopular. However, keep these facts in mind:

\begin{itemize}
\item
  A \textbf{reduction} does not equal an \textbf{elimination} of forest
  products or the forest products industry
\item
  The more reliant a community is on either export industries or the
  wood products industry, the more likely it is to suffer from higher
  levels of poverty and unemployment
\item
  Forest industry work is dangerous -- logging has the second highest
  fatality rate of all job categories
\item
  Forest industry workers are older and not being replaced by younger
  generations because of issues around job security, self-employment,
  and danger.
\item
  Plantations require regular, costly human intervention that is often
  subsidized by government programs, including repeated thinnings and
  pesticide applications.
\item
  Plantations provide substantially less water filtering, wildlife
  habitat, and carbon sequestration than natural forests.
\end{itemize}

We are not legislators, however, we can envision a reduction in forest
harvest through taking legislative action such as:

\begin{itemize}
\item
  Reducing subsidies paid to wood products companies
\item
  Reducing subsidies in state forest planting programs that clearly
  favor commercial species
\item
  Enacting more regulations on logging to improve carbon sequestration,
  water and air quality, wildlife habitat, canopy retention, and the
  availability of seed trees for regeneration.
\end{itemize}

\hypertarget{next-steps-resources}{%
\section{Next Steps \& Resources}\label{next-steps-resources}}

Do you want to act on this information, but you're not sure where to go?
We've got you covered! Here are some links\ldots{}

\end{document}
